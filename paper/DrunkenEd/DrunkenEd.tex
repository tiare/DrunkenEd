\documentclass{chi-ext}
% Please be sure that you have the dependencies (i.e., additional LaTeX packages) to compile this example.
% See http://personales.upv.es/luileito/chiext/

%% EXAMPLE BEGIN -- HOW TO OVERRIDE THE DEFAULT COPYRIGHT STRIP -- (July 22, 2013 - Paul Baumann)
% \copyrightinfo{Permission to make digital or hard copies of all or part of this work for personal or classroom use is granted without fee provided that copies are not made or distributed for profit or commercial advantage and that copies bear this notice and the full citation on the first page. Copyrights for components of this work owned by others than ACM must be honored. Abstracting with credit is permitted. To copy otherwise, or republish, to post on servers or to redistribute to lists, requires prior specific permission and/or a fee. Request permissions from permissions@acm.org. \\
% {\emph{CHI'14}}, April 26--May 1, 2014, Toronto, Canada. \\
% Copyright \copyright~2014 ACM ISBN/14/04...\$15.00. \\
% DOI string from ACM form confirmation}
%% EXAMPLE END -- HOW TO OVERRIDE THE DEFAULT COPYRIGHT STRIP -- (July 22, 2013 - Paul Baumann)

\title{Drunken Ed: A Balance Game for Public Large Screen Displays}

\newcommand{\tuaddr}{
\affaddr{Technische Universit\"at Berlin}\\
\affaddr{Stra\ss e des 17. Juni 135}\\
\affaddr{Berlin, 10623 Germany}\\
}

\numberofauthors{5}
% Notice how author names are alternately typesetted to appear ordered in 2-column format;
% i.e., the first 4 autors on the first column and the other 4 auhors on the second column.
% Actually, it's up to you to strictly adhere to this author notation.
\author{
	\alignauthor{
		\textbf{Alexander Biskupski}\\
  	\tuaddr
  	\email{alexbis@mailbox.tu-berlin.de}
  }\alignauthor{
  	\textbf{Marcel Karsten}\\
  	\tuaddr
  	\email{marcel.karsten@campus.tu-berlin.de}
  }
  \vfil
  \alignauthor{
		\textbf{Andreas R. Fender}\\
  	\tuaddr
  	\email{fenderandreas@mailbox.tu-berlin.de}
  }\alignauthor{
		\textbf{Jonas D. Willaredt}\\
  	\tuaddr
  	\email{jwill@cs.tu-berlin.de}
  }
  \vfil
	\alignauthor{
		\textbf{Tiare M. Feuchtner}\\
  	\tuaddr
  	\email{tiare.feuchtner@campus.tu-berlin.de}
  }
}

% Paper metadata (use plain text, for PDF inclusion and later re-using, if desired)
\def\plaintitle{Drunken Ed: A Balance Game for Public Large Screen Displays}
\def\plainauthor{Alexander Biskupski, Andreas Fender, Tiare Feuchtner, Marcel Karsten, Jonas Willaredt}
\def\plainkeywords{Public displays, large screen display, casual game, balance game, Kinect, gesture control}
%\def\plaingeneralterms{Documentation, Standardization}

\hypersetup{
  % Your metadata go here
  pdftitle={\plaintitle},
  pdfauthor={\plainauthor},  
  pdfkeywords={\plainkeywords},
 % pdfsubject={\plaingeneralterms},
  % Quick access to color overriding:
  %citecolor=black,
  %linkcolor=black,
  %menucolor=black,
  %urlcolor=black,
}

\usepackage{graphicx}   % for EPS use the graphics package instead
\usepackage{balance}    % useful for balancing the last columns
\usepackage{bibspacing} % save vertical space in references

\usepackage{graphicx}

\newcommand{\drunkened}{\textit{Drunken Ed}}
\newcommand{\ed}{Ed}
\newcommand{\eds}{Ed's}
\newcommand{\lbreak}{\\[0.7cm]}

\newcommand{\image}[2]{
\begin{center}
\includegraphics[scale = #1]{pictures/#2}
\end{center}
}
\newcounter{imgcounter}
\newcommand{\figimage}[4]{
\begin{figure}[h] 
\image{#1}{#2}
\caption{\textit{#3}\\#4}
\label{#2}
\end{figure}
}
\newcommand{\capimage}[3]{
\begin{center}
\includegraphics[scale = #1]{#2}\\
\begin{small}#3\end{small}
\end{center}
}

\newcommand{\fig}[2]{
\begin{figure}
  \centering
  \includegraphics[width=\linewidth]{pictures/#1}
  \caption{#2}
  \label{fig:#1}
\end{figure}
}

\begin{document}

\maketitle


\begin{abstract}
\drunkened\ is a 2D balance game that was specifically designed for public displays. We show that this casual game is well suited for public context and that camera based body tracking offers convenient interaction techniques for large screen displays. The game setting with its drunken protagonist Ed was well received by the players. This single player game uses the angle of the player's torso in relation to the ground to help Ed keep balance in a wobbling world. Ed's body pose reflects the player's pose, creating a very direct form of control. Results of our evaluation show that this form of mapping is very easy to learn and the short play sessions meet the requirements of a casual game in public environment. Furthermore we have designed our level selection menu thus, that it fulfills the double purpose of a gameplay tutorial, allowing the player to get familiar with the controls before plunging into the game.
%Do not change the page size or page settings.
\end{abstract}

\keywords{\plainkeywords}

\category{H.5.m}{Information interfaces and presentation (e.g., HCI)}{Miscellaneous}. 


% =============================================================================
\section{Introduction}
% =============================================================================
The balance game \drunkened\ was developed with the goal of creating an application for public large screen displays, which could provide some sort of entertainment. 
%and potentially raise awareness for the negative effects of drinking. 
\marginpar{
\fig{inGame2}{In game scene with Ed approaching obstacles}
\fig{storyboard1}{Storyboard with a probable interaction scenario}
\fig{paperPrototype}{Paper prototype of Drunken Ed to test interaction}
}
Games are intrinsically motivating \cite{salen2004rules} \cite{malone1981motivation}, thus it is not required that the application fulfills any further useful purpose. By designing the game for a large screen display with vision based body tracking for the controls, \drunkened\ can be easily installed in a public area, e.g. a waiting room, where it serves to entertain and shorten the wait.\\
The gameplay of \drunkened\ can be briefly described as a single player balance game. The mission is to guide Ed home safely: The players must keep balance and avoid obstacles, as depicted in \autoref{fig:inGame2}, while attempting to move right along the sidewalk as far as they can. The direct control mapping is easy to understand and helps players to relate to the protagonist more strongly.
%The starting point of the game, is the minimalistic level selection menu displaying Ed in a pub shown in \autoref{fig:screenshot2}. Here the player may experiment with the controls and make Ed stagger left and right by leaning. On the counter there are three different types of alcohol: Beer, Wine and Vodka. The player can make Ed drink one of these by positioning Ed accordingly and performing a drinking gesture. This component of choice entails a further motivation factor according to Malone \cite{malone1981motivation}. Animations and hints inform the player about his options. Upon drinking, the game starts and Ed begins his unsteady way home. The difficulty of the selected level depends on the type of drink consumed - stronger alcohol makes it more difficult for Ed to keep balance. The difficulty further increases with time. A counter informs the player about his progress. Upon losing balance and falling down, a highscore list is shown. If the player has managed a top score, he can have his picture taken to be displayed on the list.
Ed never actually reaches his home, thus this is purely an endurance game with the goal of travelling as far as possible. We hereby allow players who approach in groups to compete with their friends for the best scores. With appealing visuals we strive to provide entertainment to players as well as spectators, which has previously been discussed by O'Hara et al \cite{ohara2008understanding}.\\
%To ensure suitability for public context, we aimed to fulfill the criteria casual games as proposed by Kultima et al. \cite{kultima2009casual}


%\drunkened\ is a 2D balance game specifically designed for public displays. The player stands in front of a large display and is tracked by a Kinect. By bending his or her own upper body, the player steers a drunkyard called \textit{\ed}. The goal of the game is to walk as far as possible without falling down. While the time passes, the difficulty increases contineously. So the player has to keep the balance to not fall down, but at the same time, he or she has to hurry because it is getting harder and harder to increase the walked distance.\\
One key element of the visuals is the rotating camera: many balance games keep the world static while letting the player balance an object. In this game, the opposite is the case: it is the world around the figure, which needs to be balanced with respect to the upper body. 


%third person, but subjective

% =============================================================================
\section{Design Principles}
% =============================================================================
%Since \drunkened\ is a public display game, it has to have several properties:
\begin{itemize}\compresslist
\item It must be very easy to understand and to play. Therefore, the game is reduced to one main input (the upper body) and one simple goal (walking to the right).
\item One game session must be quite short. The game and also public display games in general are targeting people, which are actually not planning to play. Therefore, they might not want to spent to much time with a game, that they started spontaneously. Furthermore, \drunkened\ is a single player game, so players have to take turns, which is easier with short game sessions.
\item The narrative elements must be reduced to a minimum. There are no cutscenes, long texts etc. to tell the game's background. The animations, the players actions and keywords like ''distance'' give the game a humorous context.
\end{itemize}

\drunkened\ targets at the joy of failing, that is, letting \ed\ eventually tumble and fall asleep. \eds\ tumbling is not implemented as an animation, but physics based, i.e. depending on \eds\ locomotion and angular velocity. This means, that failing varies from player to player, since they can take \ed\ into individual, often awkward, sleeping postures. This also attempts to compensate the fact, that \drunkened\ is a single player game, because spectators can have fun seeing others fail.
To create a game suitable for public displays, we designed \drunkened\ with the following properties, which were partly inspired by the work of Kultima et al \cite{kultima2009casual}:
\begin{itemize}\compresslist
\item \textit{Simplicity:} The game must be very easy to understand and to play. Therefore, it is reduced to one main input (the upper body orientation) and one simple goal (walking to the right).
\item \textit{Brevity:} Games on public displays target people who are usually passing by without any intention to play, and it has been shown that interaction with public displays happens in short bursts \cite{walter2013strikeapose}. Therefore we designed our play sessions to be short to encourage turn taking among multiple players.
\item \textit{Acceptability:} This was a strongly discussed aspect, since our protagonist indulges in the pleasure of consuming alcohol, which causes him to stagger along the streets drunkenly. We have found that this situation is widely accepted and considered fun, specially among young people. Our game humorously points out the problematic consequences of drinking.
\item \textit{Accessibility:} Our physical setup with a large screen and visual tracking allows the game to be played by anybody who approaches it and does not require any prior knowledge or special abilities. However the tracking with the Kinect sets a couple of limitations, such as when dealing with occlusion, bad lighting situations or overcrowding.
\item \textit{Flexibility:} This is ensured by forgiving mistakes and gracefully recovering from situations such as when the player suddenly leaves the tracking area.
\end{itemize}

%The narrative elements must be reduced to a minimum. There are no cutscenes, long texts etc. to tell the game's background. The animations, the players actions and keywords like ''distance'' give the game a humorous context.
%\drunkened\ targets a joy of failing, that is, letting \ed\ eventually tumble and fall asleep. \eds\ tumbling is not implemented as an animation, but physics based, i.e. depending on \eds\ locomotion and angular velocity. This means, that failing varies from player to player, since they can take \ed\ into individual, often awkward, sleeping postures. This also attempts to compensate the fact, that \drunkened\ is a single player game, because spectators can have fun seeing others fail.

% =============================================================================
\section{Design Process}
% =============================================================================
\drunkened\ was created in a strongly participatory design process which entailed the creation of story boards (\autoref{fig:storyboard1}) for use in semi structured interviews and paper prototyping (\autoref{fig:paperPrototype}) for Wizard of Oz experiments.
%\subsection{Project Outline}
%After brainstorming we specified the project outline defining the setting of the installation, theme and main mechanics of the balance game and clarified the requirements. 
%Furthermore we created a fictional persona for whom we designed the game.
%\subsection{Storyboarding and Interviews}
%In order to explore possible scenarios for the deployment of our game we created story boards [picture of story board]. With these we approached three potential users and conducted semi-structured interviews to enquire about the need of such a form of entertainment in the given situation and also clarify the desirability of a  game such as \drunkened\ and the acceptability of its drinking context. From these interviews we learned that..
%At a later point we invited participants to take part in an online vote, to make sure that this controversial subject was truly well accepted. Here we offered the option of a balance game featuring a dish washer balancing plates or a sleepwalking guy. A total of x people participated with a majority of votes for \drunkened.
%\subsection{Paper Prototype and User Study}
%Prior to implementation we performed user studies with a detailed paper prototype. 
With these we researched the comprehensibility of our level selection menu, the user's readiness to learn the controls and the need for in-game hints and explanations. At this stage we made valuable observations on which we based our design decisions, such as to implement a menu, which is operable with the same mechanics as the game itself. Thus the menu serves a double purpose as a tutorial teaching the player how to play.
\subsection{Heuristic Prototype Evaluation}
\marginpar{
\fig{userPlaying}{Performing a think aloud study with users}
\fig{bendHint}{Hints are displayed to teach players the controls}
\fig{stepOnMark}{Inviting the player to step onto the mark to play}
\fig{titleScreen}{Initial screen while no player is detected}
}
After developing a first interactive prototype, we  analyzed our game regarding Nielsen’s 10 Usability Heuristics \cite{nielsen1995usability} and added two further principles targeting the aspects of \textit{motivation} and a \textit{steep learning curve} required for games in public context.
%Here we wish to explain our additional principles.
%
%\begin{enumerate}
	%\item Motivation: All difficulty levels are available to play, so an advanced player does not need to stagger through all the easier stages first to reach a sufficiently challenging level. The highscores displayed above the level selection should provide the necessary motive to start another game and improve your score.
	%\item Learning curve (Additional Principle): A player can improve his performance and score, if he plays more often. 
%\end{enumerate}
%Changes were made according to the results of this evaluation, as to the visibility of system status by adding a countdown at level start and including a highscore list for motivation.
A think aloud study was performed thereafter with several users of varying expertise (\autoref{fig:userPlaying}), which lead to further improvements of the game according to Nielsen's heuristics. The following subsection will address the challenges we faced and the design decision resulting thereof.
\subsection{Challenges}
\begin{itemize}
	\item Steep learning curve due to brief interaction sessions: To show the player \textit{how} to interact as quickly as possible, we designed a game with natural input gestures and direct mapping. In addition we implemented hints with both animated text and icons as proposed by Walter et al. \cite{walter2013strikeapose}, demonstrating the possible actions, if no valid input is made for a certain time (\autoref{fig:bendHint}).
\item Creating a level selection menu for gesture based interaction: This process lead from a button interface based on hand selection with dwell times, to an approach with full body interaction and direct gesture mapping. In plain terms: The spatial menu representation may be navigated by leaning, thus making Ed stumble left or right. A selection is possible once Ed holds the drink of choice in hand and can be triggered by performing a drinking gesture, which makes Ed empty the drink.
\item Invisible limits of the tracking area: A simple solution would be to mark the interaction area on the floor in front of the screen. This however imposes certain requirements on the installation location, which we wished to avoid. Therefore we implemented a warning screen to be displayed when the user leaves the optimal tracking region for error prevention. This screen shows the optimal spot marked with a red \"x\" and the player's relative position, with an arrow guiding her back (\autoref{fig:stepOnMark}).
\end{itemize}
\subsection{Implementation}
The game was implemented using OpenGL for the graphics and SimpleOpenNI to process input from the Kinect sensor. The players are represented by the rag doll \ed, who reflects their body pose and forms the fixed axis around which the world begins to rotate. This is unusual in balance games, where most often the world remains fixed. \eds\ tumbling is physics based, i.e. it depends on \eds\ locomotion and angular velocity. This means that each player fails by bringing \ed\ into individual and often awkward sleeping postures.

% =============================================================================
\section{Game Play and Mechanics}
% =============================================================================
While no player is present a bar is shown with three blackboards displaying the highscores for the respective levels (\autoref{fig:titleScreen}). A call to action invites passers-by to enter the tracking area, upon which \ed\ appears. By bending to one side the player can move \ed\ along the bar to choose a level by selecting one of three alcoholic drinks which are placed on the counter: beer, wine or vodka (\autoref{fig:mainMenu}). The difficulty of each level corresponds to the type of alcohol consumed - beer signifies the easiest level while vodka is the most challenging, causing the world to rotate faster and more uncontrollably. The choice is confirmed by performing an expressive drinking gesture.\\
At this point, the main game starts with the objective to make \ed\ walk as far as  possible. \ed\ walks in an uncontrolled manner always following his center of mass, resembling the typical accentuated movement of a drunken person. 
\marginpar{
\fig{mainMenu}{Selecting a level and difficulty by choosing a drink}
\fig{fallDown1}{The game ends when Ed falls down}
\fig{gameOver2}{Game over screen after reaching a highscore}
\fig{swingArms}{Ed spreading and swinging his arms as a warning}
}
\eds\ upper body has the same orientation as that of the player and creates a fixed axis around which the world rotates. Bending \eds\ upper body shifts his center of mass, making him stumble in a certain direction. The closer \ed\ bends towards the ground, the faster he walks, and the player must compensate the world's rotation to keep \ed\ from falling down. This happens as soon as the angle between the upper body and the floor becomes too small. In addition \ed\ stumbles if he goes too fast, or trips over obstacles that lie in his path. For the latter, a short hop makes \ed\ leap into the air.\\
The game is over once \ed\ falls down (\autoref{fig:fallDown1}) and the distance traveled is shown as the player's score. Players with a highscore may take a picture of themselves for the highscore list above the bar (see \autoref{fig:gameOver2}), which is once again triggered by the drinking gesture. To avoid taking the picture, the player can wait for the countdown to end, or simply leave the playing area. The game then returns to the level selection menu.\\
%If a player should accidentally leave the tracking area, a hint invites her to step back onto a mark shown on the screen (see \autoref{fig:stepOnMark}).\\
During gameplay \eds\ arms play an important role to convey information to the player. In the menu the arms are controllable by the player, while in the main game \eds\ arms hang down idly emphasizing the loss of physical control. Furthermore this contributes to the players orientation and allow to estimate \eds\ movement, because his arms always point towards the ground and center of gravity. They also have an important feedback role which can be seen in \autoref{fig:swingArms}: If \ed\ walks too fast he begins to flail and if he is close to falling, his arms start to swing as if trying to balance. Players quickly interpreted these actions as alarming indicators.
%As long as no player stands in front of the screen, a bar is shown together with three blackboards containing the highscores. A hint invites the player to step onto the mark, which is located in front of the screen. When a player is detected, the main protagonist \ed appears. The player can now get used to the controls, without having to worry about keeping balance yet. At the same time, the player chooses the difficulty, encoded as three different alcoholic drinks: beer, wine and wodka. By bending him or herself, the player can move \ed\ to the respective drink. The user confirms his or her choice by performing a drinking gesture. This expressive gesture is explained by hints as proposed by Walter et al. \cite{walter2013strikeapose}\lbreak

At this point, the main game starts. The player must make \ed\ walk as far to the right as possible. \ed\ does not walk in a controlled manner, but always follows his center of mass. This resembles the typical accentuated movement of a drunken person. The player bends his or her upper body to control \eds\ upper body, which in turn shifts his center of mass. \ed\ gathers speed, the more he bends. The core mechanic is the rotating world: The upper body of \ed\ has the same orientation as that of the player with respect to the screen, but not with respect to the rotating world around \ed. Therefore, the player must compensate the world's rotation to not tumble, which happens as soon as the angle between the upper body and the floor becomes too small. Additionally, \ed\ stumbles if he goes too fast.\\
The higher the difficulty, the faster and more uncontrollably the world rotates, which makes it harder to keep balance and make \ed\ proceed to the right at the same time.\lbreak

\ed\ will eventually fall down. The game over overlay appears and the distance \ed\ walked until falling down is presented to the player as his score. If the player got a top three score, he or she can take a picture of him or herself to appear in the highscore list of the bar. The player does so by doing the drink gesture, in order that the picture captures him or her in a drinking posture. If the player does not want to take a picture, he or she has to wait for some seconds without doing anything or to leave the play area. Afterwards, the game restarts and the player gets back to the difficulty selection.\lbreak

During gameplay, the arms of \ed\ play an important role: In the difficulty selection, the arms are controllable by the players arms. However, in the main game, resembling typical cartoony postures of drunkyards, the arms are saggy, pointing straight towards the floor. Firstly, this emphasizes the loss of physical control, because the players arm movements are ignored now. Furthermore, they contribute to the players orientation, because \eds\ arms being aligned with the upper body mean a neutral posture without movement. When they are not aligned, the angle helps estimating the movement. In addition, the arms have an important feedback role: if \ed\ is getting too fast, they start to flail. If \ed\ is about to overbend, they start to swing. Players quickly understood those actions as alarming indicators.
% This also attempts to compensate the fact, that \drunkened\ is a single player game, because spectators can have fun seeing others fail.
% =============================================================================
\section{Conclusion}
% =============================================================================
%The game is still under development. Right now, placeholder graphics are replaced and new obstacle elements are added to enhance the gameplay.
Creating games for public displays brings many new challenges in game design. These range from novel input techniques and effectively teaching unaccustomed game mechanics, to catering for strongly varying degrees of expertise and interest. With our work we wish to give valuable insights into some of these aspects and propose the direct mapping of body poses as effective controls for gesture based interaction with public displays. 
%Furthermore we wish to point out the potential of public displays as carriers of messages, such as to create awareness for problems as for instance the effects of alcohol abuse, which we wish to address more strongly in future development.
%In future we plan to enhance the gameplay and add more variety to our game in form of new obstacles and differing scenery. 

% =============================================================================
\section{Acknowledgments}
% =============================================================================
Special thanks go to our project supervisors J\"org M\"uller and Robert Walter for their help during development. Furthermore we wish to thank all volunteers who participated in our interviews, experiments and testing.
%%--------------------REMOVE FOLLOWING SECTIONS---------------------------------
%% =============================================================================
%\section{Copyright}
%% =============================================================================
%For publications in the CHI Extended Abstracts, copyright remains with the author.  
%The publication is not considered an archival publication; however, it does go into the ACM Digital Library. 
%Because you retain copyright, as the author you are free to use this material as you like, including submitting a paper based on this work to other conferences or journals.  
%Authors grant unrestricted permission for ACM to publish the accepted submission in the CHI Extended Abstracts without additional consideration or remuneration.
%
%% =============================================================================
%\section{Text formatting}
%% =============================================================================
%Please use an 8.5-point Verdana font, or other sans serifs font as close as possible in appearance to Verdana in which these guidelines have been set. 
%Arial 9-point font is a reasonable substitute for Verdana as it has a similar x-height. 
%Please use serif or non-proportional fonts only for special purposes, such as distinguishing source code text.
%Additionally, here is an example of footnoted text.\footnote{Use footnotes sparingly, if at all.}
%As stated in the footnote, footnotes should rarely be used.
%
%\subsection{Language, style, and content}
%% -----------------------------------------------------------------------------
%The written and spoken language of SIGCHI is English. 
%Spelling and punctuation may use any dialect of English (e.g., British, Canadian, US, etc.) provided this is done consistently. 
%Hyphenation is optional. 
%To ensure suitability for an international audience, please pay attention to the following:
%
%\begin{itemize}\compresslist
%\item 	
%Write in a straightforward style. 
%Use simple sentence structure. 
%Try to avoid long sentences and complex sentence structures. 
%Use semicolons carefully.
%\item 	
%Use common and basic vocabulary (e.g., use the word ``unusual" rather than the word ``arcane").
%\item 	
%Briefly define or explain all technical terms. 
%The terminology common to your practice/discipline may be different in other design practices/disciplines.
%\item 	
%Spell out all acronyms the first time they are used in your text. 
%For example, ``World Wide Web (WWW)".
%\item 	
%Explain local references (e.g., not everyone knows all city names in a particular country).
%\item 	
%Explain ``insider" comments. 
%Ensure that your whole audience understands any reference whose meaning you do not describe (e.g., do not assume that everyone has used a Macintosh or a particular application).
%\item 	
%Explain colloquial language and puns. 
%Understanding phrases like ``red herring" requires a cultural knowledge of English. 
%Humor and irony are difficult to translate.
%\item 	
%Use unambiguous forms for culturally localized concepts, such as times, dates, currencies and numbers (e.g., ``1-5-97" or ``5/1/97" may mean 5 January or 1 May, and ``seven o'clock" may mean 7:00 am or 19:00).
%\item 	
%Be careful with the use of gender-specific pronouns (he, she) and other gender-specific words (chairman, manpower, man-months). 
%Use inclusive language (e.g., she or he, they, chair, staff, staff-hours, person-years) that is gender-neutral. 
%If necessary, you may be able to use ``he" and ``she" in alternating sentences, so that the two genders occur equally often~\cite{Schwartz95}. 
%\end{itemize}
%
%
%% =============================================================================
%\section{Figures}
%% =============================================================================
%The examples on this and following pages should help you get a feel for how screen-shots and other figures should be placed in the template. 
%Be sure to make images large enough so the important details are legible and clear.
%
%\begin{figure}
  %\centering
  %\includegraphics[width=\linewidth]{DrunkenEd.jpg}
  %\caption{Insert a caption below each figure.}
  %\label{fig:DrunkenEd}
%\end{figure}
%
%Your document may use color figures, which are included in the page limit; the figures must be usable when printed in black and white.
%You can use the \LaTeX's \texttt{marginpar} command to insert figures in the (right) margin side of the document (see \autoref{fig:marginparsample}).
%
%
%% =============================================================================
%\section{References and Citations}
%% =============================================================================
%Use a numbered list of references at the end of the article, ordered alphabetically by first author, and referenced by numbers in brackets \cite{Anderson92,Klemmer02,Mather00,Zellweger01}
%For papers from conference proceedings, include the title of the paper and an abbreviated name of the conference (e.g., for Interact 2003 proceedings, use Proc. Interact 2003). 
%Do not include the location of the conference or the exact date; do include the page numbers if available. 
%See the examples of citations at the end of this document. 
%
%Your references should be published materials accessible to the public.  
%Internal technical reports may be cited only if they are easily accessible (i.e., you provide the address for obtaining the report within your citation) and may be obtained by any reader for a nominal fee.  
%Proprietary information may not be cited. 
%Private communications should be acknowledged in the main text, not referenced  (e.g., [Robertson, personal communication]).
%
%% =============================================================================
%\section{Accessibility}
%% =============================================================================
%The Executive Council of SIGCHI has committed to making SIGCHI conferences more inclusive for researchers, practitioners, and educators with disabilities. As a part of this goal, the all authors are asked to work on improving the accessibility of their submissions. Specifically, we encourage authors to carry out the following five steps:
%\begin{enumerate}
	%\item Add alternative text to all figures
	%\item Mark table headings
	%\item Add tags to the PDF
	%\item Verify the default language
	%\item Set the tab order to ``Use Document Structure''
%\end{enumerate}
%Unfortunately good tools do not yet exist to create tagged PDF files from Latex. LaTeX users will need to carry out all of the above steps in the PDF directly using Adobe Acrobat, after the PDF has been generated.
 %
%For more information and links to instructions and resources, please see:
%{\url{http://chi2014.acm.org/authors/guide-to-an-accessible-submission}}.
%
%% =============================================================================
%\section{Producing and testing PDF files}
%% =============================================================================
%We recommend that you produce a PDF version of your submission well before the final deadline. 
%Besides making sure that you are able to produce a PDF, you will need to check that (a) the length of the file remains within the submission category's page limit, (b) the PDF file size is 4 megabytes or less, and (c) the file can be read and printed using Adobe Acrobat Reader. 
%Test your PDF file by viewing or printing it with the same software we will use when we receive it, Adobe Acrobat Reader Version 7. 
%This is widely available at no cost from~\cite{Acrobat7}.  
%Note that most reviewers will use a North American/European version of Acrobat reader, which cannot handle documents containing non-North American or non-European fonts (e.g. Asian fonts).  
%Please therefore do not use Asian fonts, and verify this by testing with a North American/European Acrobat reader (obtainable as above). Something as minor as including a space or punctuation character in a two-byte font can render a file unreadable.
%
%
%% =============================================================================
%\section{Dummy text}
%% =============================================================================
%Lorem ipsum dolor sit amet, consectetur adipiscing elit. Duis ut eros semper lectus vehicula elementum. Vestibulum ante ipsum primis in faucibus orci luctus et ultrices posuere cubilia Curae; Aliquam quis mi sapien. Suspendisse potenti. Mauris ultrices euismod velit sed dictum. Nullam auctor, nulla tincidunt dapibus suscipit, velit leo convallis metus, vel commodo libero erat in dolor. In laoreet porttitor ligula, porta blandit lectus consequat quis. 
%
%Nam ut eros dui. Mauris volutpat elit metus, euismod pellentesque purus. In hac habitasse platea dictumst. Nullam consectetur lacinia interdum. Sed nec blandit nisi. Proin in nulla purus, sit amet iaculis tortor. Ut dapibus pellentesque nulla in interdum. Nunc at velit felis. Curabitur euismod neque eu orci varius in pharetra sem interdum. Morbi in mauris ac risus iaculis dapibus id in magna. Class aptent taciti sociosqu ad litora torquent per conubia nostra, per inceptos himenaeos.
%
%\marginpar{
%\begin{figure}
  %\begin{center}
  %\includegraphics[width=\marginparwidth]{DrunkenEd.jpg}
  %\caption{A marginal figure.}
  %\label{fig:marginparsample}
  %\end{center}  
%\end{figure}
%}
%Aliquam consectetur quam sed odio varius vitae rhoncus urna fermentum. Phasellus viverra diam non justo porttitor varius. Integer ultrices accumsan lectus eget mollis. Nulla et leo sit amet nunc ornare rutrum sit amet ac dui. Cras vehicula accumsan purus nec fermentum. Mauris viverra condimentum metus, ut posuere quam laoreet nec. Phasellus massa tellus, ullamcorper nec porta sed, aliquet vitae nulla. Phasellus non tortor mauris. Cras ullamcorper egestas erat, vel rutrum elit viverra a. Donec in nisl ut est facilisis blandit. Quisque congue accumsan risus, ut venenatis magna vulputate vel. Nam commodo sapien vel mauris adipiscing nec dictum quam congue. Phasellus tempor vestibulum nisl quis blandit. Nullam condimentum auctor nibh, quis elementum libero tristique.
%
%
%
%\section{Acknowledgements}
%We thank all DUX 2003 publications support and staff who wrote this document originally and allowed us to modify it for this conference.
%This template was based on Manas Tungare's \texttt{chi.cls}, and rewritten by Luis A. Leiva.
%
%\section{References format}
%References must be the same font size as other body text.
%% REFERENCES FORMAT
%% References must be the same font size as other body text.

\balance
\bibliographystyle{acm-sigchi}
\bibliography{DrunkenEd}

\end{document}