\documentclass{article}

\usepackage{graphicx}

\newcommand{\drunkened}{\textit{Drunken Ed}}
\newcommand{\ed}{Ed}
\newcommand{\eds}{Ed's}
\newcommand{\lbreak}{\\[0.7cm]}

\newcommand{\image}[2]{
\begin{center}
\includegraphics[scale = #1]{pictures/#2}
\end{center}
}
\newcounter{imgcounter}
\newcommand{\figimage}[4]{
\begin{figure}[h] 
\image{#1}{#2}
\caption{\textit{#3}\\#4}
\label{#2}
\end{figure}
}
\newcommand{\capimage}[3]{
\begin{center}
\includegraphics[scale = #1]{#2}\\
\begin{small}#3\end{small}
\end{center}
}

\begin{document}

\section*{Drunken Ed - Instruction manual}

$ $\\

\subsection*{Prerequisits}

\begin{itemize}
\item Microsoft Kinect
\item Java Runtime Environment (1.7 or later recommended)
\end{itemize}

\subsection*{Starting the game}

At first, connect the Kinect to the PC. Afterwards, execute \textit{drunken\_ed.jar}. Do not execute the jar without having a Kinect connected.

\subsection*{How to play}

When starting the jar, the title screen appears. To play the game, position yourself in front of the Kinect with a distance of about two meters. \ed\ appears. To control \ed, lean yourself to the left or to the right, respectively. It might take a short while, until the controls are fully calibrated. Move \ed\ to a drink and select it by doing a drink gesture, i.e. moving your right hand to your mouth. After a countdown, the main game starts. Try to move \ed\ to the right as far as possible without him falling down, again by leaning to the left or to the right respectively. Try to keep \eds\ balance and do not let him move to fast. When \ed\ falls down, the game is over. If your distance is a top three score, you can take a picture of yourself by doing the drink gesture. If you do not want to take a picture, wait some seconds. The score and the picture (or a placeholder) will appear in the highscore list of the main menu. If you have not done a highscore, just do the drink gesture or wait for some seconds. The game will then restart.

\end{document}